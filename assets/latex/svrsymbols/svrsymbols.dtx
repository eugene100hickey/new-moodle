%% \CharacterTable
%%  {Upper-case    \A\B\C\D\E\F\G\H\I\J\K\L\M\N\O\P\Q\R\S\T\U\V\W\X\Y\Z
%%   Lower-case    \a\b\c\d\e\f\g\h\i\j\k\l\m\n\o\p\q\r\s\t\u\v\w\x\y\z
%%   Digits        \0\1\2\3\4\5\6\7\8\9
%%   Exclamation   \!     Double quote  \"     Hash (number) \#
%%   Dollar        \$     Percent       \%     Ampersand     \&
%%   Acute accent  \'     Left paren    \(     Right paren   \)
%%   Asterisk      \*     Plus          \+     Comma         \,
%%   Minus         \-     Point         \.     Solidus       \/
%%   Colon         \:     Semicolon     \;     Less than     \<
%%   Equals        \=     Greater than  \>     Question mark \?
%%   Commercial at \@     Left bracket  \[     Backslash     \\
%%   Right bracket \]     Circumflex    \^     Underscore    \_
%%   Grave accent  \`     Left brace    \{     Vertical bar  \|
%%   Right brace   \}     Tilde         \~}
%%
%\iffalse
%
% (c) Copyright 2015-2019 Pablo Garcia Risueno, Apostolos Syropoulos, 
%                         and Natalia Verges
% This program can be redistributed and/or modified under the 
% terms of the LaTeX Project Public License Distributed from 
% http://www.latex-project.org/lppl.txt; either
% version 1.3c of the License, or any later version.
%  
% This work has the LPPL maintenance status `maintained'.
%
% Please report errors or suggestions for improvement to
%    
%    Apostolos Syropoulos  (asyropoulos@yahoo.com)
%
%\fi
% \CheckSum{344}
% \iffalse This is a Metacomment
%
%<svrsymbols, >\ProvidesFile{svrsymbols.sty}
%
%<svrsymbols, > [2019/02/12 v2.0b Package `svrsymbols.sty']
%
%    \begin{macrocode}
%<*driver>
\documentclass{ltxdoc}
\GetFileInfo{svrsymbols.drv}
\usepackage{xltxtra}
\usepackage{svrsymbols}
\usepackage{fullpage}
\usepackage{longtable}
\begin{document}
\setmainfont[Mapping=tex-text]{Universal Modern}
\setmonofont{UM Typewriter}
\setsansfont[Mapping=tex-text]{GFS Neohellenic}
   \DocInput{svrsymbols.dtx}    
\end{document}
%</driver>
%    \end{macrocode}
% \fi
%\MakeShortVerb{\|}
%\StopEventually{}
%\title{The \textsf{svrsymbols} \LaTeX\ Package:\\New ideograms for Physics}
%\author{Pablo Garc{\'\i}a Risue{\~n}o\\ Humboldt Universit\"at\\ zu Berlin, Germany\\  
%\and Apostolos Syropoulos\\ Xanthi, Greece\\ \texttt{asyropoulos@yahoo.com} 
%\and Nat\`alia Verg\'es\\ Besal\'u (Girona), Spain}
% \date{2019/02/12}
%\maketitle
% \begin{abstract}
% The \textsf{svrsymbols} package is a \LaTeX\ interface to the \textsf{SVRsymbols}
% font. The glyphs of this font are ideograms that have been designed for use in 
% Physics texts. Some symbols are standard and some are entirely new.
%\end{abstract}
% 
% \section{Introduction - Usage}
% Ideograms are present in most communication codes that are in daily use. Examples of ideograms are the digits 
% 0,1,\ldots,9, mathematical symbols (like $+$, $-$, $\in$, $\sqrt{}$, etc.), emoticons, traffic signs or commercial logos. 
% In English there are symbols that represent words (e.g., think of the symbols @, \$, and \&). In addition,
% there are symbols that are ubiquitous in certain languages (e.g., Chinese, Korean, and Japanese). Nowadays,
% the current ``corpus'' of modern languages that are written with an alphabet (e.g., Spanish and Greek) include
% certain ideograms and pictograms to enhance and simplify communication (e.g., think of smileys). 
%
%\input svrsymbols.tex
%
% Physics employs the language of mathematics to express ideas and facts. Nevertheless, in Physics 
% certain letters and symbols have reserved meaning. The \textsf{SVRsymbols}\footnote{This version of
% the package is bundled with an OpenType version of the font so that people who use LibreOffice can
% also use it in their documents.} font contain some new ideograms 
% for use in Physics. The symbols  have been designed so to be intuitive, easy to identify and to remember. 
% The package \textsf{svrsymbols} currently has no options and provides an interface to the font. In 
% particular, it defines commands that work only in math mode and provide access to the various glyphs of 
% the \textsf{SVRsymbols} font. These commands as well as the symbols each command corresponds to are shown 
% in Table~\ref{tbl1}. In addition there are three more commands that can be used get size-variants of the 
% $\assumption$ symbol:
% \begin{displaymath}
%  \bigassumption\rightarrow\mbox{\texttt{\char`\\bigassumption}},\;  
%  \biggassumption\rightarrow\mbox{\texttt{\char`\\biggassumption}},\;
%  \Bigassumption\rightarrow\mbox{\texttt{\char`\\Bigassumption}}.
% \end{displaymath}
% \section{The Source code}
% The first part of the code is the identification part.
%    \begin{macrocode}
%<*svrsymbols>
\NeedsTeXFormat{LaTeX2e}
\ProvidesPackage{svrsymbols}
          [2019/02/12 v.2.0b, New Symbols for Physics.]
%    \end{macrocode}
% The commands that follow define commands according to the NFSS necessary to access the font that contains
% the various glyphs. First we define a new font family and then the various variants. Since there are no
% variants, the commands use the ``default'' font.
%    \begin{macrocode}
\DeclareFontFamily{OML}{svr}{}
\DeclareFontShape{OML}{svr}{m}{it}{
   <-> SVRsymbols
}{}
\DeclareFontShape{OML}{svr}{b}{it}{
   <-> SVRsymbols
}{}
\DeclareFontShape{OML}{svr}{m}{sl}{<->ssub * svr/m/it}{}
\DeclareFontShape{OML}{svr}{bx}{it}{<->ssub * svr/b/it}{}
\DeclareFontShape{OML}{svr}{b}{sl}{<->ssub * svr/b/it}{}
\DeclareFontShape{OML}{svr}{bx}{sl}{<->ssub * svr/b/sl}{}
\DeclareSymbolFont{svrsymbols}{OML}{svr}{m}{it}
\SetSymbolFont{svrsymbols}{bold}{OML}{svr}{b}{it}
%    \end{macrocode}
% The commands that follow are the glyph access commands. Let us stress again that these commands can
% be used only in math mode.
%    \begin{macrocode}
\DeclareMathSymbol{\method}{\mathord}{svrsymbols}{`A}
\DeclareMathSymbol{\orbit}{\mathord}{svrsymbols}{`B}
\DeclareMathSymbol{\atom}{\mathord}{svrsymbols}{`C}
\DeclareMathSymbol{\antiproton}{\mathord}{svrsymbols}{`D}
\DeclareMathSymbol{\antiquark}{\mathord}{svrsymbols}{`E}
\DeclareMathSymbol{\antiquarkb}{\mathord}{svrsymbols}{`F}
\DeclareMathSymbol{\antiquarkc}{\mathord}{svrsymbols}{`G}
\DeclareMathSymbol{\antiquarkd}{\mathord}{svrsymbols}{`H}
\DeclareMathSymbol{\antiquarks}{\mathord}{svrsymbols}{`I}
\DeclareMathSymbol{\antiquarkt}{\mathord}{svrsymbols}{`J}
\DeclareMathSymbol{\antiquarku}{\mathord}{svrsymbols}{`K}
\DeclareMathSymbol{\varphoton}{\mathord}{svrsymbols}{`L}
\DeclareMathSymbol{\antineutrino}{\mathord}{svrsymbols}{`M}
\DeclareMathSymbol{\neutrino}{\mathord}{svrsymbols}{`N}
\DeclareMathSymbol{\quark}{\mathord}{svrsymbols}{`O}
\DeclareMathSymbol{\quarkb}{\mathord}{svrsymbols}{`P}
\DeclareMathSymbol{\quarkc}{\mathord}{svrsymbols}{`Q}
\DeclareMathSymbol{\quarkd}{\mathord}{svrsymbols}{`R}
\DeclareMathSymbol{\quarks}{\mathord}{svrsymbols}{`S}
\DeclareMathSymbol{\quarkt}{\mathord}{svrsymbols}{`T}
\DeclareMathSymbol{\quarku}{\mathord}{svrsymbols}{`U}
\DeclareMathSymbol{\dipole}{\mathord}{svrsymbols}{`V}
\DeclareMathSymbol{\spindown}{\mathord}{svrsymbols}{`Z}
\DeclareMathSymbol{\electron}{\mathord}{svrsymbols}{`a}
\DeclareMathSymbol{\svrphoton}{\mathord}{svrsymbols}{`b}
\DeclareMathSymbol{\fermiDistrib}{\mathord}{svrsymbols}{`c}
\DeclareMathSymbol{\proton}{\mathord}{svrsymbols}{`d}
\DeclareMathSymbol{\nucleus}{\mathord}{svrsymbols}{`e}
\DeclareMathSymbol{\ion}{\mathord}{svrsymbols}{`f}
\DeclareMathSymbol{\neutron}{\mathord}{svrsymbols}{`g}
\DeclareMathSymbol{\hole}{\mathord}{svrsymbols}{`h}
\DeclareMathSymbol{\exciton}{\mathord}{svrsymbols}{`i}
\DeclareMathSymbol{\phonon}{\mathord}{svrsymbols}{`j}
\DeclareMathSymbol{\polaron}{\mathord}{svrsymbols}{`k}
\DeclareMathSymbol{\reference}{\mathord}{svrsymbols}{`l}
\DeclareMathSymbol{\positron}{\mathord}{svrsymbols}{`m}
\DeclareMathSymbol{\antiproton}{\mathord}{svrsymbols}{`n}
\DeclareMathSymbol{\spinup}{\mathord}{svrsymbols}{`o}
\DeclareMathSymbol{\plasmon}{\mathord}{svrsymbols}{`p}
\DeclareMathSymbol{\errorsym}{\mathord}{svrsymbols}{`q}
\DeclareMathSymbol{\water}{\mathord}{svrsymbols}{`r}
\DeclareMathSymbol{\graphene}{\mathord}{svrsymbols}{`s}
\DeclareMathSymbol{\solid}{\mathord}{svrsymbols}{`t}
\DeclareMathSymbol{\assumption}{\mathord}{svrsymbols}{`u}
\DeclareMathSymbol{\bigassumption}{\mathord}{svrsymbols}{"C8}
\DeclareMathSymbol{\biggassumption}{\mathord}{svrsymbols}{"C9}
\DeclareMathSymbol{\Bigassumption}{\mathord}{svrsymbols}{"CA}
\DeclareMathSymbol{\experimentalsym}{\mathord}{svrsymbols}{`v}
\DeclareMathSymbol{\antimuon}{\mathord}{svrsymbols}{`w}
\DeclareMathSymbol{\muon}{\mathord}{svrsymbols}{`x}
\DeclareMathSymbol{\antineutron}{\mathord}{svrsymbols}{`y}
\DeclareMathSymbol{\surface}{\mathord}{svrsymbols}{`z}
\DeclareMathSymbol{\fermion}{\mathord}{svrsymbols}{"A1}
\DeclareMathSymbol{\externalsym}{\mathord}{svrsymbols}{"A2}
\DeclareMathSymbol{\internalsym}{\mathord}{svrsymbols}{"A3}
\DeclareMathSymbol{\maxwellDistrib}{\mathord}{svrsymbols}{"A4}
\DeclareMathSymbol{\resistivity}{\mathord}{svrsymbols}{"A5}
\DeclareMathSymbol{\boseDistrib}{\mathord}{svrsymbols}{"A6}
\DeclareMathSymbol{\boson}{\mathord}{svrsymbols}{"A7}
\DeclareMathSymbol{\spin}{\mathord}{svrsymbols}{"C0}
\DeclareMathSymbol{\polariton}{\mathord}{svrsymbols}{"C1}
\DeclareMathSymbol{\conductivity}{\mathord}{svrsymbols}{"C2}
\DeclareMathSymbol{\bond}{\mathord}{svrsymbols}{"C3}
\DeclareMathSymbol{\covbond}{\mathord}{svrsymbols}{"C4}
\DeclareMathSymbol{\doublecovbond}{\mathord}{svrsymbols}{"C5}
\DeclareMathSymbol{\triplecovbond}{\mathord}{svrsymbols}{"C6}
\DeclareMathSymbol{\metalbond}{\mathord}{svrsymbols}{"C7}
\DeclareMathSymbol{\hbond}{\mathord}{svrsymbols}{"CB}
\DeclareMathSymbol{\svrexample}{\mathord}{svrsymbols}{"CC}
\DeclareMathSymbol{\magnon}{\mathord}{svrsymbols}{"CD}
\DeclareMathSymbol{\ionicbond}{\mathord}{svrsymbols}{"CE}
\DeclareMathSymbol{\tachyon}{\mathord}{svrsymbols}{"CF}
\DeclareMathSymbol{\adsorbent}{\mathord}{svrsymbols}{"D0}
\DeclareMathSymbol{\adsorbate}{\mathord}{svrsymbols}{"D1}
\DeclareMathSymbol{\anyon}{\mathord}{svrsymbols}{"D2}
\DeclareMathSymbol{\interaction}{\mathord}{svrsymbols}{"D3}
\DeclareMathSymbol{\quadrupole}{\mathord}{svrsymbols}{"D4} 
\DeclareMathSymbol{\protein}{\mathord}{svrsymbols}{"D5} 
\DeclareMathSymbol{\tauleptonplus}{\mathord}{svrsymbols}{"D6}
\DeclareMathSymbol{\tauleptonminus}{\mathord}{svrsymbols}{"D7}
\DeclareMathSymbol{\Bmesonplus}{\mathord}{svrsymbols}{"D9}
\DeclareMathSymbol{\Bmesonminus}{\mathord}{svrsymbols}{"DA}
\DeclareMathSymbol{\Bmesonnull}{\mathord}{svrsymbols}{"DB}
\DeclareMathSymbol{\Dmesonplus}{\mathord}{svrsymbols}{"DC}
\DeclareMathSymbol{\Dmesonminus}{\mathord}{svrsymbols}{"DD}
\DeclareMathSymbol{\Dmesonnull}{\mathord}{svrsymbols}{"DE}
\DeclareMathSymbol{\Tmesonplus}{\mathord}{svrsymbols}{"DF}
\DeclareMathSymbol{\Tmesonminus}{\mathord}{svrsymbols}{"E0}
\DeclareMathSymbol{\Tmesonnull}{\mathord}{svrsymbols}{"E1}
\DeclareMathSymbol{\Upsilonmeson}{\mathord}{svrsymbols}{"E2}
\DeclareMathSymbol{\phimeson}{\mathord}{svrsymbols}{"E3}
\DeclareMathSymbol{\phimesonnull}{\mathord}{svrsymbols}{"E4}
\DeclareMathSymbol{\rhomesonplus}{\mathord}{svrsymbols}{"E5}
\DeclareMathSymbol{\rhomesonminus}{\mathord}{svrsymbols}{"E6}
\DeclareMathSymbol{\rhomesonnull}{\mathord}{svrsymbols}{"E7}
\DeclareMathSymbol{\etameson}{\mathord}{svrsymbols}{"E8}
\DeclareMathSymbol{\etamesonprime}{\mathord}{svrsymbols}{"E9}
\DeclareMathSymbol{\pionplus}{\mathord}{svrsymbols}{"EA}
\DeclareMathSymbol{\pionminus}{\mathord}{svrsymbols}{"EB}
\DeclareMathSymbol{\pionnull}{\mathord}{svrsymbols}{"EC}
\DeclareMathSymbol{\Kaonplus}{\mathord}{svrsymbols}{"ED}
\DeclareMathSymbol{\Kaonminus}{\mathord}{svrsymbols}{"EE}
\DeclareMathSymbol{\Kaonnull}{\mathord}{svrsymbols}{"EF}
\DeclareMathSymbol{\Gluon}{\mathord}{svrsymbols}{"F0}
\DeclareMathSymbol{\Higgsboson}{\mathord}{svrsymbols}{"F1}
\DeclareMathSymbol{\Wbosonplus}{\mathord}{svrsymbols}{"F2}
\DeclareMathSymbol{\Wbosonminus}{\mathord}{svrsymbols}{"F3}
\DeclareMathSymbol{\Wboson}{\mathord}{svrsymbols}{"F4}
\DeclareMathSymbol{\Zboson}{\mathord}{svrsymbols}{"F5}
\DeclareMathSymbol{\Jpsimeson}{\mathord}{svrsymbols}{"F6}
\DeclareMathSymbol{\graviton}{\mathord}{svrsymbols}{"F7}
%</svrsymbols>
%    \end{macrocode}
% \Finale
